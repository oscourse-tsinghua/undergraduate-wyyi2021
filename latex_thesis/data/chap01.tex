% !TeX root = ../thuthesis-example.tex

\chapter{绪论}

\section{研究背景}

交通是人类社会进步过程中不可缺少的一环。随着交通系统基础设施的逐步完善,人们在依赖便捷交通的同时也产生了更多的需求;面对出行安全、道路规划、环境污染等诸多潜在问题,智能交通系统的概念诞生了,旨在将基础设施与先进的信息技术结合,利用交通环境中产生的大量信息进行数字化、智能化的管理。

车载自组网(VANET,vehicle ad-hoc networks)是智能交通系统框架中的重要概念之一。车载自组网是专门为车辆间通信而设计的自组织网络,其基本思想是在一定通信范围内的车辆可以共享自己的地理位置、行驶状态以及车载传感器感知的数据,并自发地连接建立起一个移动的网络。通过这些数据的传递,可以实现路况查询、事故预警等功能,从而提高人们的出行体验和整体的交通效率。

然而,考虑到车载自组网通信对象陌生、时空特性变化快的特点,相互通信的邻近车辆不一定是完全可信的,特别是当道路上存在意图在车载网中进行恶意攻击的车辆时,很可能会导致更加严重的安全问题。因此,如何对接入网络的车辆进行科学的可信度评估,成为了保障车载网安全性的一个重要问题。对所有车辆进行公开的信誉评估,可以使车辆在网络通信的过程中自行决定是否信任对方提供的信息,从而在系统层面上确保交通状况能得到真实、准确的反映,并且行驶车辆能够及时地接收到这些信息。

区块链技术起源于比特币,本质上是一种分布式的共享账本和数据库,具有去中心化、不可篡改、可追溯、集体维护、公开透明、时序性强等特点。对于没有网络中心、流动性强的车载网来说,区块链技术的应用可以很好地提供一个车辆间协作的便捷途径,为位置验证、消息传递等功能提供安全性的保障。

因此,本文希望将结合区块链技术与车载自组网,并为车载网应用中的交互参与者提供量化的、可靠的信誉信息。本文基于车载网上车辆的交互信息,设计了一个评估车辆信誉的算法,并基于区块链实现了一套车载网络中的车辆信誉评估系统,与车辆的位置验证、消息传播等应用活动进行结合,在地图中展示车辆行驶信息、交互结果与信誉评估。

\section{相关工作}

\subsection{区块链与智能合约}
区块链技术起源于比特币,是近年来逐渐兴起的一种互联网应用模式,其中涉及到点对点传输、分布式管理、去中心化、共识机制、加密算法等技术,在金融领域大获成功之后,逐渐扩展至公共管理、物联网、医疗等领域。在汽车领域,文献\cite{blockchain0}利用区块链为智能车辆实现了一种安全的交易体系结构,将数据与区块链交易分离,可以在车辆保险费用缴纳等方面进行应用。

智能合约是区块链的核心技术之一,为区块链去中心化的特点提供了技术上的重要支持。智能合约从被触发到运行完成的流程可合并为一次原子操作,它按照获得全网共识的计算规则自动执行合约内容,在不需要第三方介入的情况下完成安全可靠的合约执行,具有安全可靠、不易篡改的特点。目前,以太坊智能合约的应用较为广泛。文献\cite{blockchaineth}将以太坊账户作为车载网参与者的个人账户,通过定制智能合约实现可查询交通违法记录、缴纳各项费用的自管理系统。

\subsection{车辆位置验证}

位置验证是证明一个人在某个时间处于某个确定的地理位置的一个数字签名\cite{p2ppof}。车载自组网中要求车辆提供具有时效性、准确性的位置验证,以便于更多的车辆行驶信息能有效地应用于更多基于位置的服务( Location-Based Service,LBS)。

目前,多数车辆位置验证采用的是依赖于网络基础架构的集中式验证,主要通过无线接入点和蜂巢网络为用户提供位置验证。\cite{alice, Lu2016Privacy},这种验证方式在服务器上统一存储用户提交的位置验证申请,以及对其进行确认所需的信息。然而考虑到车载网的流动性,集中式验证往往难以满足车辆对于实时性的要求;同时,作为绝对权威的中心服务器一旦遭受攻击,所有接入网络的车辆都将面临严重的安全问题。

不依赖网络基础架构的验证方式通常利用邻近车辆或车上的移动设备获取有效的位置验证。该验证方式要求参与设备具有一定的短距离通信功能(比如蓝牙)\cite{2013Toward},通过带有私钥签名的点对点传输完成验证,并在互联网中传播验证结果;特别地,对于分布式的位置验证系统,传播后的位置验证结果被写入区块链中\cite{p2ppof}。

\subsection{车载网信誉管理系统}

中心化的车辆信誉管理系统通常使用一个服务器作为完全可信的权威机构,车辆之间的反馈被统一提交并在服务器上进行信誉的计算\cite{reputation};这种利用单一中心节点的方式随着智能交通系统的快速发展,已经很难承担起大量车辆对低延迟数据获取的需求。对于去中心化的信誉管理系统,信誉计算的压力被分配到了车载网中的各个节点上,通常在各车辆\cite{ephemeral}或是路侧单元(Roadside Unit,RSU)\cite{2017Distributed}内部执行;但是在本地计算的信誉值很难保证可靠性,同时由于RSU作为分布在室外的基础设施,比较容易受到攻击和损害。\cite{2018Blockchain}提出了一种基于区块链技术的去中心化车辆信誉管理方式,利用RSU进行数据处理和信誉维护,并通过区块链结构保证数据的同步和安全性,降低单个RSU故障对系统带来的影响;同时,该文章提出了一种基于消息传播的信誉计算方法,通过车辆与消息描述事件之间的距离完成消息评价。文献\cite{rss}和\cite{density}同样基于消息进行信誉更新,前者以车辆间的信号接受强度(RSS)作为评价指标,后者则考虑每条消息影响到的车辆规模。
文献\cite{v2v}中的车辆信誉受车辆间数据信息交易结果的影响,对于每一台车辆维护一个朴素贝叶斯网络,在其叶节点中更新该车辆各个交易身份对应的不同信誉值。文献\cite{edge}提出了在物联网的边缘计算中,完成时间更近的互动结果在用户声望计算中占有更大的权重。文献\cite{lwq}实现了一个基于区块链的车辆修正与信誉评估系统,仅基于车辆自身提供的数据(如行驶速度、位置修正结果)进行信誉计算。
在已有的相关工作中,多数文章只针对单一的影响因子进行信誉值的计算,而没有综合考虑车辆的多种行为。本文在参考文献\cite{lwq}的基本结构的同时,重新设计了车辆信誉评估算法,综合考量多种车辆交互过程中的影响因素。

\section{论文的研究内容及贡献}

本文设计了一套车载自组网中车辆信誉评估的算法,并基于该算法实现了一套信誉评估系统。

\begin{enumerate}
    \item 基于车辆的行驶状态和车辆在网络中的交互,提出一套信誉评估算法,使得车辆的可信度能够得到较为全面、真实的反映,同时利用以太坊智能合约实现该算法。
    \item 设计实验,通过模拟数据对算法中的参数特征进行测试、分析与调优,评估参数选取对算法科学性、稳定性的影响。
    \item 将智能合约与网页端结合,实现车载网信誉评估的原型系统,使得车辆行驶数据与信誉值在区块链上得到分布式存储,同时在浏览器端进行车辆交互的模拟、信誉评估结果的展示;同时利用该原型系统对模拟数据及真实数据进行测试,验证测试结果与算法设计目标的对应情况。
\end{enumerate}
