% !TeX root = ../thuthesis-example.tex

\chapter{车辆信誉评估算法的设计}

\section{插图}

图片通常在 \env{figure} 环境中使用 \cs{includegraphics} 插入,如图~\ref{fig:example} 的源代码。
建议矢量图片使用 PDF 格式,比如数据可视化的绘图;
照片应使用 JPG 格式;
其他的栅格图应使用无损的 PNG 格式。
注意,LaTeX 不支持 TIFF 格式;EPS 格式已经过时。

\begin{figure}
  \centering
  \includegraphics[width=0.6\linewidth]{example-image-a.pdf}
  \caption{示例图片}
  \label{fig:example}
\end{figure}

若图或表中有附注,采用英文小写字母顺序编号,附注写在图或表的下方。
% LaTeX 传统上一般将附注的内容同图表的标题写在一起,形成很长的一段文字。

如果一个图由两个或两个以上分图组成时,各分图分别以 (a)、(b)、(c)...... 作为图序,并须有分图题。
推荐使用 \pkg{subcaption} 宏包来处理, 比如图~\ref{fig:subfig-a} 和图~\ref{fig:subfig-b}。

\begin{figure}
  \centering
  \subcaptionbox{分图 A\label{fig:subfig-a}}
    {\includegraphics[width=0.45\linewidth]{example-image-a.pdf}}
  \subcaptionbox{分图 B\label{fig:subfig-b}}
    {\includegraphics[width=0.45\linewidth]{example-image-b.pdf}}
  \caption{多个分图的示例}
  \label{fig:multi-image}
\end{figure}



\section{表格}

表应具有自明性。为使表格简洁易读,尽可能采用三线表,如表~\ref{tab:three-line}。
三条线可以使用 \pkg{booktabs} 宏包提供的命令生成。

\begin{table}
  \centering
  \caption{三线表示例}
  \begin{tabular}{ll}
    \toprule
    文件名          & 描述                         \\
    \midrule
    thuthesis.dtx   & 模板的源文件,包括文档和注释 \\
    thuthesis.cls   & 模板文件                     \\
    thuthesis-*.bst & BibTeX 参考文献表样式文件    \\
    thuthesis-*.bbx & BibLaTeX 参考文献表样式文件  \\
    thuthesis-*.cbx & BibLaTeX 引用样式文件        \\
    \bottomrule
  \end{tabular}
  \label{tab:three-line}
\end{table}

表格如果有附注,尤其是需要在表格中进行标注时,可以使用 \pkg{threeparttable} 宏包。
研究生要求使用英文小写字母 a、b、c……顺序编号,本科生使用圈码 ①、②、③……编号。

\begin{table}
  \centering
  \begin{threeparttable}[c]
    \caption{带附注的表格示例}
    \label{tab:three-part-table}
    \begin{tabular}{ll}
      \toprule
      文件名                 & 描述                         \\
      \midrule
      thuthesis.dtx\tnote{a} & 模板的源文件,包括文档和注释 \\
      thuthesis.cls\tnote{b} & 模板文件                     \\
      thuthesis-*.bst        & BibTeX 参考文献表样式文件    \\
      thuthesis-*.bbx        & BibLaTeX 参考文献表样式文件  \\
      thuthesis-*.cbx        & BibLaTeX 引用样式文件        \\
      \bottomrule
    \end{tabular}
    \begin{tablenotes}
      \item [a] 可以通过 xelatex 编译生成模板的使用说明文档;
        使用 xetex 编译 \file{thuthesis.ins} 时则会从 \file{.dtx} 中去除掉文档和注释,得到精简的 \file{.cls} 文件。
      \item [b] 更新模板时,一定要记得编译生成 \file{.cls} 文件,否则编译论文时载入的依然是旧版的模板。
    \end{tablenotes}
  \end{threeparttable}
\end{table}

如某个表需要转页接排,可以使用 \pkg{longtable} 宏包,需要在随后的各页上重复表的编号。
编号后跟表题(可省略)和“(续)”,置于表上方。续表均应重复表头。

\begin{longtable}{cccc}
    \caption{跨页长表格的表题} \\
    \toprule
    表头 1 & 表头 2 & 表头 3 & 表头 4 \\
    \midrule
  \endfirsthead
    \caption[]{跨页长表格的表题(续)} \\
    \toprule
    表头 1 & 表头 2 & 表头 3 & 表头 4 \\
    \midrule
  \endhead
    \bottomrule
  \endfoot
  Row 1  & & & \\
  Row 2  & & & \\
  Row 3  & & & \\
  Row 4  & & & \\
  Row 5  & & & \\
  Row 6  & & & \\
  Row 7  & & & \\
  Row 8  & & & \\
  Row 9  & & & \\
  Row 10 & & & \\
  Row 11 & & & \\
  Row 12 & & & \\
  Row 13 & & & \\
  Row 14 & & & \\
  Row 15 & & & \\
  Row 16 & & & \\
  Row 17 & & & \\
  Row 18 & & & \\
  Row 19 & & & \\
  Row 20 & & & \\
  Row 21 & & & \\
  Row 22 & & & \\
  Row 23 & & & \\
  Row 24 & & & \\
  Row 25 & & & \\
  Row 26 & & & \\
  Row 27 & & & \\
  Row 28 & & & \\
  Row 29 & & & \\
  Row 30 & & & \\
  Row 31 & & & \\
  Row 32 & & & \\
  Row 33 & & & \\
  Row 34 & & & \\
  Row 35 & & & \\
  Row 36 & & & \\
  Row 37 & & & \\
  Row 38 & & & \\
  Row 39 & & & \\
  Row 40 & & & \\
\end{longtable}



\section{算法}

算法环境可以使用 \pkg{algorithms} 或者 \pkg{algorithm2e} 宏包。

\renewcommand{\algorithmicrequire}{\textbf{输入:}\unskip}
\renewcommand{\algorithmicensure}{\textbf{输出:}\unskip}

\begin{algorithm}
  \caption{Calculate $y = x^n$}
  \label{alg1}
  \small
  \begin{algorithmic}
    \REQUIRE $n \geq 0 \vee x \neq 0$
    \ENSURE $y = x^n$

    \STATE $y \leftarrow 1$
    \IF{$n < 0$}
       \STATE $X \leftarrow 1 / x$
       \STATE $N \leftarrow -n$
    \ELSE
       \STATE $X \leftarrow x$
       \STATE $N \leftarrow n$
    \ENDIF

    \WHILE{$N \neq 0$}
      \IF{$N$ is even}
        \STATE $X \leftarrow X \times X$
        \STATE $N \leftarrow N / 2$
      \ELSE[$N$ is odd]
        \STATE $y \leftarrow y \times X$
        \STATE $N \leftarrow N - 1$
      \ENDIF
    \ENDWHILE
  \end{algorithmic}
\end{algorithm}

% !TeX root = ../thuthesis-example.tex

\chapter{数学符号和公式}

\section{数学符号}

中文论文的数学符号默认遵循 GB/T 3102.11—1993《物理科学和技术中使用的数学符号》
\footnote{原 GB 3102.11—1993,自 2017 年 3 月 23 日起,该标准转为推荐性标准。}。
该标准参照采纳 ISO 31-11:1992 \footnote{目前已更新为 ISO 80000-2:2019。},
但是与 \TeX{} 默认的美国数学学会(AMS)的符号习惯有所区别。
具体地来说主要有以下差异:
\begin{enumerate}
  \item 大写希腊字母默认为斜体,如
    \begin{equation*}
      \Gamma \Delta \Theta \Lambda \Xi \Pi \Sigma \Upsilon \Phi \Psi \Omega.
    \end{equation*}
    注意有限增量符号 $\increment$ 固定使用正体,模板提供了 \cs{increment} 命令。
  \item 小于等于号和大于等于号使用倾斜的字形 $\le$、$\ge$。
  \item 积分号使用正体,比如 $\int$、$\oint$。
  \item 行间公式积分号的上下限位于积分号的上下两端,比如
    \begin{equation*}
      \int_a^b f(x) \dif x.
    \end{equation*}
    行内公式为了版面的美观,统一居右侧,如 $\int_a^b f(x) \dif x$ 。
  \item
    偏微分符号 $\partial$ 使用正体。
  \item
    省略号 \cs{dots} 按照中文的习惯固定居中,比如
    \begin{equation*}
      1, 2, \dots, n \quad 1 + 2 + \dots + n.
    \end{equation*}
  \item
    实部 $\Re$ 和虚部 $\Im$ 的字体使用罗马体。
\end{enumerate}

以上数学符号样式的差异可以在模板中统一设置。
另外国标还有一些与 AMS 不同的符号使用习惯,需要用户在写作时进行处理:
\begin{enumerate}
  \item 数学常数和特殊函数名用正体,如
    \begin{equation*}
      \uppi = 3.14\dots; \quad
      \symup{i}^2 = -1; \quad
      \symup{e} = \lim_{n \to \infty} \left( 1 + \frac{1}{n} \right)^n.
    \end{equation*}
  \item 微分号使用正体,比如 $\dif y / \dif x$。
  \item 向量、矩阵和张量用粗斜体(\cs{symbf}),如 $\symbf{x}$、$\symbf{\Sigma}$、$\symbfsf{T}$。
  \item 自然对数用 $\ln x$ 不用 $\log x$。
\end{enumerate}


英文论文的数学符号使用 \TeX{} 默认的样式。
如果有必要,也可以通过设置 \verb|math-style| 选择数学符号样式。

关于量和单位推荐使用
\href{http://mirrors.ctan.org/macros/latex/contrib/siunitx/siunitx.pdf}{\pkg{siunitx}}
宏包,
可以方便地处理希腊字母以及数字与单位之间的空白,
比如:
\SI{6.4e6}{m},
\SI{9}{\micro\meter},
\si{kg.m.s^{-1}},
\SIrange{10}{20}{\degreeCelsius}。



\section{数学公式}

数学公式可以使用 \env{equation} 和 \env{equation*} 环境。
注意数学公式的引用应前后带括号,建议使用 \cs{eqref} 命令,比如式 \eqref{eq:example}。
\begin{equation}
  \frac{1}{2 \uppi \symup{i}} \int_\gamma f = \sum_{k=1}^m n(\gamma; a_k) \mathscr{R}(f; a_k)
  \label{eq:example}
\end{equation}
注意公式编号的引用应含有圆括号,可以使用 \cs{eqref} 命令。

多行公式尽可能在“=”处对齐,推荐使用 \env{align} 环境。
\begin{align}
  a & = b + c + d + e \\
    & = f + g
\end{align}



\section{数学定理}

定理环境的格式可以使用 \pkg{amsthm} 或者 \pkg{ntheorem} 宏包配置。
用户在导言区载入这两者之一后,模板会自动配置 \env{thoerem}、\env{proof} 等环境。

\begin{theorem}[Lindeberg--Lévy 中心极限定理]
  设随机变量 $X_1, X_2, \dots, X_n$ 独立同分布, 且具有期望 $\mu$ 和有限的方差 $\sigma^2 \ne 0$,
  记 $\bar{X}_n = \frac{1}{n} \sum_{i+1}^n X_i$,则
  \begin{equation}
    \lim_{n \to \infty} P \left(\frac{\sqrt{n} \left( \bar{X}_n - \mu \right)}{\sigma} \le z \right) = \Phi(z),
  \end{equation}
  其中 $\Phi(z)$ 是标准正态分布的分布函数。
\end{theorem}
\begin{proof}
  Trivial.
\end{proof}

同时模板还提供了 \env{assumption}、\env{definition}、\env{proposition}、
\env{lemma}、\env{theorem}、\env{axiom}、\env{corollary}、\env{exercise}、
\env{example}、\env{remar}、\env{problem}、\env{conjecture} 这些相关的环境。

% !TeX root = ../thuthesis-example.tex

\chapter{引用文献的标注}

模板支持 BibTeX 和 BibLaTeX 两种方式处理参考文献。
下文主要介绍 BibTeX 配合 \pkg{natbib} 宏包的主要使用方法。


\section{顺序编码制}

在顺序编码制下,默认的 \cs{cite} 命令同 \cs{citep} 一样,序号置于方括号中,
引文页码会放在括号外。
统一处引用的连续序号会自动用短横线连接。

\thusetup{
  cite-style = super,
}
\begin{tabular}{l@{\quad$\Rightarrow$\quad}l}
  \verb|\cite{zhangkun1994}|               & \cite{zhangkun1994}               \\
  \verb|\citet{zhangkun1994}|              & \citet{zhangkun1994}              \\
  \verb|\citep{zhangkun1994}|              & \citep{zhangkun1994}              \\
  \verb|\cite[42]{zhangkun1994}|           & \cite[42]{zhangkun1994}           \\
  \verb|\cite{zhangkun1994,zhukezhen1973}| & \cite{zhangkun1994,zhukezhen1973} \\
\end{tabular}


也可以取消上标格式,将数字序号作为文字的一部分。
建议全文统一使用相同的格式。

\thusetup{
  cite-style = inline,
}
\begin{tabular}{l@{\quad$\Rightarrow$\quad}l}
  \verb|\cite{zhangkun1994}|               & \cite{zhangkun1994}               \\
  \verb|\citet{zhangkun1994}|              & \citet{zhangkun1994}              \\
  \verb|\citep{zhangkun1994}|              & \citep{zhangkun1994}              \\
  \verb|\cite[42]{zhangkun1994}|           & \cite[42]{zhangkun1994}           \\
  \verb|\cite{zhangkun1994,zhukezhen1973}| & \cite{zhangkun1994,zhukezhen1973} \\
\end{tabular}



\section{著者-出版年制}

著者-出版年制下的 \cs{cite} 跟 \cs{citet} 一样。

\thusetup{
  cite-style = author-year,
}
\begin{tabular}{l@{\quad$\Rightarrow$\quad}l}
  \verb|\cite{zhangkun1994}|                & \cite{zhangkun1994}                \\
  \verb|\citet{zhangkun1994}|               & \citet{zhangkun1994}               \\
  \verb|\citep{zhangkun1994}|               & \citep{zhangkun1994}               \\
  \verb|\cite[42]{zhangkun1994}|            & \cite[42]{zhangkun1994}            \\
  \verb|\citep{zhangkun1994,zhukezhen1973}| & \citep{zhangkun1994,zhukezhen1973} \\
\end{tabular}

\vskip 2ex
\thusetup{
  cite-style = super,
}
注意,引文参考文献的每条都要在正文中标注
\cite{zhangkun1994,zhukezhen1973,dupont1974bone,zhengkaiqing1987,%
  jiangxizhou1980,jianduju1994,merkt1995rotational,mellinger1996laser,%
  bixon1996dynamics,mahui1995,carlson1981two,taylor1983scanning,%
  taylor1981study,shimizu1983laser,atkinson1982experimental,%
  kusch1975perturbations,guangxi1993,huosini1989guwu,wangfuzhi1865songlun,%
  zhaoyaodong1998xinshidai,biaozhunhua2002tushu,chubanzhuanye2004,%
  who1970factors,peebles2001probability,baishunong1998zhiwu,%
  weinstein1974pathogenic,hanjiren1985lun,dizhi1936dizhi,%
  tushuguan1957tushuguanxue,aaas1883science,fugang2000fengsha,%
  xiaoyu2001chubanye,oclc2000about,scitor2000project%
}。
