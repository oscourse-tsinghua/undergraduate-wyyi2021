\chapter{结论}
车载自组网通过车辆行驶数据的共享与传播为出行便利与交通安全提供技术上的支持,但是由于其流动性强、结构开放等特点,邻近车辆之间的通信信息并不总是真实可信的。针对这一问题,本文提出了基于车辆交互结果的车辆信誉值算法,利用车辆间位置验证、消息传播的通讯结果对车辆的信誉值进行实时更新,使得车辆的可信度能够得到合理的量化体现,同时对于车载网中可能出现的攻击、欺骗行为具有一定的反应能力。

除此之外,本文引入区块链技术与智能合约的部署,与网页端相结合,实现了一套车辆信誉评估的原型系统,在完成车辆用户验证、行驶状态追踪的基础上,于目标车辆与邻近车辆之间模拟位置验证及消息传播的交互结果,并在区块链上执行信誉算法,在网页端地图中进行交互结果及信誉变化的可视化展示。

最后,通过对指定交通场景的模拟以及真实数据的处理,本文对原型系统进行了测试,对算法中的参数特征进行了评估,验证了系统在信誉计算功能上的科学性。测试表明,该系统对用户活跃度、交互失败结果具有较高的敏感度,在实际道路环境中的信誉变化与算法设计目标有较好的吻合度。在未来的工作中,需要采集更多实际数据对该系统进行完善的测试,并考虑将系统扩展至面向多台车辆的信誉计算,对系统的性能及资源消耗进行进一步优化。